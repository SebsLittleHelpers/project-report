\subsection{Description of the optimisation}
Scala allows for multiple constructors per classes which is not the case for
JavaScript. Up until now, Scala.js solved that problem by only having the
JavaScript constructor initialize the class fields with the zero of their type
and then have an "init" method per constructor which actually does what its
corresponding Scala constructor does. This means that currently if you want to
create an instance of type A the outputted JavaScript code will first create an
instance of type A with the "empty" constructor and then call the init method
on it. \\
But in most cases, a Scala class doesn't have more than one constructor and if
that class is effectively final (has no subclasses with instances), otherwise
there would be issues in the subclasses with the call to the optimized parent
constructor. In those cases, it would be possible to get rid of the init method
and directly "inline" the Scala constructor instructions in the JavaScript
constructor, right after initializing the field s with zeros of their type.
This optimization would reduce the amount of code generated and reduce the
number of method call executed thus improving both the file size and the
running time. The code snippet \ref{lst:ctorOptScala} shows the Scala code for
an optimizable class and an example of call site; code example
\ref{lst:ctorOptJSNo} shows the current way of defining such class; finally,
code snippet \ref{lst:ctorOptJS} shows the generated code we want to build when
inlining the constructor. (these examples do not show all the generated
JavaScript code, only the relevant parts)

\begin{lstlisting}[language=scala,caption=Example Scala Code.,
label={lst:ctorOptScala}]
class A(message: String) {
    def tell(): Unit = println(message)
}

...

val someA = new A(someString)

...

\end{lstlisting}
\begin{lstlisting}[language=javascript,caption=Generated JavaScript without
optimisation., label={lst:ctorOptJSNo}]
/** @constructor */
function $c_LA() {
  $c_O.call(this);
  this.message$2 = null
}

/* Skipping prototype shenanigans ... */

$c_LA.prototype.init___T = (function(message) {
  this.message$2 = message;
  $c_O.prototype.init___.call(this);
  return this
});

...

var someA = new $c_LA().init___T(someString);

...

\end{lstlisting}
%$

\begin{lstlisting}[language=javascript,caption=Generated JavaScript with
constructor optimisation., label={lst:ctorOptJS}]
/** @constructor */
function $c_LA(message) {
  $c_O.call(this);
  this.message$2 = null;
  this.message$2 = message;
  $c_O.prototype.init___.call(this);
}

/* Skipping prototype shenanigans ... */

/* No init method! */

...

var someA = new $c_LA(someString);

...

\end{lstlisting}

\subsection{Implementation Details}
This implies changes in the emitting phase, when the Compiler IR is desugared
into JavaScript IR. The first step is to go through all classes and find those
who only have one constructor, we count the methods of the class where their
name is a constructor name (starting with "init\_\_\_"), and are effectively
final, we look at the parent of each classes that has instances, those parent
are not effectively final. This allows us to build a set with all the classes
that uses the constructor Optimisation. Then we have to have a different
behavior for the following point in code emitting:
\begin{itemize}
\item when building the constructor of a JS Class; if it uses the optimisation
we have to fetch the "init" method from the methods of the class, change the
parameter list and add the statement from the fetched method without its return
statement as it is now part of the constructor.
\item when generating all the method code we have to filter out the "init"
method which is no longer relevant and should not appear in the generated code.
\item when desugaring a function, each time we encounter a call site to a new
we have to check wether to call the constructor only or the "empty" constructor
and then an "init" method.
\end{itemize}

\subsection{Additional remarks}
While implementing this optimisation we faced different issues which required
restrictions or rethinking of our initial or intermediate solution. This
section describe the nature of the problem and shows how we restricted or
changed the implementation of our optimisation.

\subsubsection{Hard-coded classes}
Some of the Scala.js environment is specified in an hardcoded file named
\emph{scalajsenv.js} (or an equivalent file for EcmaScript 6 Strong Mode) most
of the classes instantiated in that file (e.g. \emph{jl\_NullPointerException})
do not always have all their constructor present in the generated code,
therefor they won't always be optimised and it is not possible to hardcode
either version of the constructor call site. We chose to exclude those classes
in the form of a blacklist when computing the set of classes which will be
optimised. Only one class instantiated in that file was always optimised and
its call site was changed in the hardcoded file to match the optimised form.
\subsubsection{Exported classes}
Optimising exported classes is problematic because it requires the code
importing our class to have knowledge whether the class has been optimised or
not. For this reason, we do not optimise exported classes.

\subsubsection{EcmaScript 6 Strong Mode}
Scala doesn't have any restriction regarding the instruction you can execute in
a constructor whereas the specification for EcmaScript 6 Strong Mode only
allows a limited set of instruction to be done inside a constructor. As such if
the selected output mode is EcmaScript 6 Strong Mode then we never do the
optimisation and keep the "init" method.

\subsubsection{Incremental compiler}
Last but not least, the naive implementation described above doesn't take into
account the fact that the Scala.js compiler works in an incremental manner; a
class/method will go through the full compiler pipeline only if it has been
modified. But with the optimisation described above, we might optimise class A
(taken from example \ref{lst:ctorOptScala}) in a run and then, during the next
compile run our program might have a new class B extending our class A; which
means we will no longer optimise the class A. This will break everything as all
the statement where we instantiate the class A might be in a class/method where
nothing has changed since the last run and as such the callsite to A
constructor will not be changed back to the non-optimized version.
To correct this issue we used the same idea that was already implemented on the
\emph{OptimizerCore}: keep track of the methods each time they ask to know if a
class uses the constructor optimisation. This is done through a new class: the
\emph{IncClassEmitter}; it creates and control the ScalaJSClassEmitter and
keeps the decision on whether or not a class uses the constructor optimisation
while keeping a map of each classes have been asked about and a list of who
asked. This enable us, at the beginning of each run, to compute the Set of
classes for which their constructor optimisation changed (compared to the
previous run) and then invalidate all methods who asked for a class that
changed. Then we can proceed with the habitual progression of the compiler
pipeline and all the code parts that needed an update will be reprocessed.
