Scala allows for multiple constructors per classes which is not the case for JavaScript.
Up until now, ScalaJS solved that problem by only having the JavaScript constructor initialize the class fields with the zero of their type and then have a method \emph{init} per constructor which actually does what its corresponding Scala constructor does. This means that currently if you want to create an instance of type A the outputted JavaScript code will first create an instance of type A with the "empty" constructor and then call the init method on it. \\
But in most cases, a Scala class doesn't have more than one constructor and if that class is effectively final (has no subclasses with instances), otherwise there would be issues in the subclasses with the call to the optimized parent constructor. In those cases, it would be possible to get rid of the init method and directly "inline" the Scala constructor instructions in the JavaScript constructor, right after initializing the field s with zeros of their type. This optimisation would reduce the amount of code generated and reduce the number of method call executed thus improving both the file size and the running time. Example 1 shows the current way of defining such class and an example of call site; example 2 shows the generated code we want to build when inlining the constructor.
%TODO set the example number accordingly to code listings
%TODO add code snippets to show the old and new way to proceed
This 