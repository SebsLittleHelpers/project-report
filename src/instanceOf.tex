\subsection{Motivation}
Scala, like most object oriented languages has a construct that allows
programmatic membership tests. Idiomatic Scala depends heavily on this feature,
especially though the use of pattern matching.
The javascript virtual machine has support for membership test (the
$instanceof$ operator), but it's behaviour is semantically different from
Scala's $isInstanceOf[T]$ method, therefore forcing Scala.js to provide some
runtime support. \\
The current implementation relies on a field $\$classData$ present at
runtime in the prototype of each class. This field contains among other things,
information about the class' ancestors as well as weather it is an instance of
the ArrayClass and it's depth. The ancestors field is a javascript object where
the keys are the name of the ancestors and the values are all set to $1$, so
that an instance test is just a lookup in the object to check whether the key
is present. 

\begin{lstlisting}[language=javascript,caption=Old membership test,
label={lst:oldIsInstanceOf}]

function $is_F0(obj) {
  return (!(!((obj && obj.$classData) && 
            obj.$classData.ancestors.F0)));
}

\end{lstlisting}
This can be quite slow for several reasons, the main on being that it needs to
lookup fields in 2 objects ($ancestors$, and then $F0$).\\

\subsection{Description of the optimisation}
To improve the efficiency of the membership test, we assign a unique numerical
identifier to each class in the program. We then generate for each class an
integer array filled with all 0 except for all the indices of the class'
subtypes. The membership test is then reduced to an array lookup, which is much
more efficient than an field lookup in an object. 

\begin{lstlisting}[language=javascript,caption=New membership test,
label={lst:newIsInstanceOf}]

var $Is_F0 = [1,0,0,0,1,...,0,0,1];

function $is_F0(obj) {
  return (!(!(obj && $Is_F0[obj.$typeTag])));
}

\end{lstlisting}

\subsection{Further optimisations}
The main problem with the previous implementation, is that it increases
dramatically the size of the generated javascript program. To attenuate this
effect, we extend the Scala.js runtime support to include a
$expandSubtypeArray$ function that will construct the array at runtime from a
list of intervals.\\

\begin{lstlisting}[language=javascript,caption=Compressed subtype array,
label={lst:expandSubtypeArray}]

var $expandSubtypeArray = function(tags){
  var expanded = [];
  var len = 0;
  for (var i = 0; i < tags.length; i++) {
    var interval = tags[i];
    var start = interval[0];
    var end = interval[1];
    while(len < start){ len = expanded.push(0); };
    while(len <= end){ len = expanded.push(1); };
  }
  return expanded;
};

var $Is_F0$ = $expandSubtypeArray([[13, 13], [21, 34]]));
\end{lstlisting}

We also try to use relative numbering \cite{relativeNumbering} to map classes
to their tags, such that in most cases, the subtype array doesn't even need to
be generated. In those cases, the membership test is reduced to a mere few
comparisons.

\begin{lstlisting}[language=javascript,
caption=Constant folding of subtype array, 
label={lst:noSubtypeArray}]

function $is_F0(obj) {
  return (!(!(obj && ((obj.$typeTag === 13) || 
      ((obj.$typeTag >= 21) && (obj.$typeTag <= 34))))))
}
  
\end{lstlisting}

\subsection{Additional remarks}
The scheme presented above works well for static membership tests, when all the
classes are known at compile time. It doesn't work for dynamic membership
tests, or when new classes are created at runtime. Fortunately, it can be
extended to work in these cases too, as presented in this section

\subsubsection{Array types}
The only types that can be created at runtime in the Scala.js environment are
the array types. To handle array, the previous implementation stored the
baseType as well as the depth of the array in the $\$classsData$ field. We
extend our previous implementation to accommodate array types by the following
encoding of the $\$tagData$ field: The first bit is reserved to discriminate
array types from other types, this makes it easy to check for array types as
all of their tags are negatives. The next 8 bits are reserved for the depth of
the array. This is sufficient since the maximum depth allowed for
multidimensional arrays in javascript is 256. The 23 bits left are then
used to encode the array's component type tag. This reduces the maximum number
of compile-time types that we can encode to roughly 8 millions. However, since
a high number of types also means a very large javascript program, reaching
this limit is very unlikely before the program reaches a problematic size.

\subsubsection{Dynamic membership tests}
Because the transformation presented above will not work for dynamic membership
tests, we cannot discard the $\$classData$ object and its $ancestors$ field
entirely. However, since dynamic tests are a lot less frequent, we can migrate
all those object to a program-global data structure indexed by each class' tag.
Furthermore, since there is a one-to-one mapping from classes to tags, we can
compress the ancestors object to an array of corresponding tags. As we see
later, this reduces the size of the output program enough to counter balance
the increase caused by the subtype arrays.



