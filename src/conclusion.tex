The first goal of the optimisations we implemented for this project was to
improve the performance of the generated code. We managed to accomplish that
goal but we were quite impressed by the reduction in generated file size.
The only results that we would like to see are the improvements in speed and
code size when combining both the isInstanceOf and constructor inlining
optimisation! But this should happen soon when both optimisation will be merged
together.

\subsection{Going further}
Because Scala.js version 0.6.6 is really close and implementation
tweaks are still needed, the isInstanceOf and constructor inlining optimisation
are yet to be merged into Scala.js. The logical next step is to do the required
tweaks to both optimisation so they can be part of the next Scala.js release.\\
Regarding the constructor inlining, we could optimise classes effectively
final with more than just one constructor. One way to do so would be to have
to inline the primary constructor, just as we currently do when there is
only one constructor, and have the auxiliary constructors as static methods
calling the primary constructor and then executing its instructions. But more
thoughts are needed to find out if it's the best way to proceed.\\
Regarding the isInstanceOf optimisation, the first thing to do is to implement
incrementality as it suffers from the same issues as the constructor inlining
optimisation. There is also room for another big optimisation. Code generated
from Scala tends to obtain most it's calls to $isInstanceOf$ from pattern
matching. By detecting chains of if statement over $isInstanceOf$ calls in the
optimiser, we could potentially rewrite them into native javascript $switch$
statement over the object's tag, eliminating a large amount of method calls.