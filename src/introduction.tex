Nowadays, a huge part of our computer utilisation is through a web browser and
web apps.
Developping such apps is most commonly done in JavaScript as it is pretty much
universally supported. Scala.js allows you to develop those kind of programs
without the hassle of using JavaScript directly, with the type safety of your
regular Scala and still have a flawless interoperability with the many existing
JavaScript libraries \cite{scalajsInterop}.
Since its conception, Scala.js has seen a lot of optimization and rewriting
done in the sole purpose to increase its performances and reduce the size of
its generated code.
In this project, we were tasked to first implement or port a benchmarking
framework to Scala.js to have a reliable way to compare two implementations of
the same feature and be able to tell which one is the best in a relatively
effortless manner.
After that we took a deep dive in the Scala.js tools and were tasked to
implement several optimisation to the generated javascript code. We first
implemented the constant folding for the binary operator \emph{String\_+},
described in chapter 3. Then we rewrote the instance checking with a tag system
and eventually we worked on the inlining constructor in specific cases.